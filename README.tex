% Options for packages loaded elsewhere
\PassOptionsToPackage{unicode}{hyperref}
\PassOptionsToPackage{hyphens}{url}
%
\documentclass[
]{article}
\usepackage{amsmath,amssymb}
\usepackage{iftex}
\ifPDFTeX
  \usepackage[T1]{fontenc}
  \usepackage[utf8]{inputenc}
  \usepackage{textcomp} % provide euro and other symbols
\else % if luatex or xetex
  \usepackage{unicode-math} % this also loads fontspec
  \defaultfontfeatures{Scale=MatchLowercase}
  \defaultfontfeatures[\rmfamily]{Ligatures=TeX,Scale=1}
\fi
\usepackage{lmodern}
\ifPDFTeX\else
  % xetex/luatex font selection
\fi
% Use upquote if available, for straight quotes in verbatim environments
\IfFileExists{upquote.sty}{\usepackage{upquote}}{}
\IfFileExists{microtype.sty}{% use microtype if available
  \usepackage[]{microtype}
  \UseMicrotypeSet[protrusion]{basicmath} % disable protrusion for tt fonts
}{}
\makeatletter
\@ifundefined{KOMAClassName}{% if non-KOMA class
  \IfFileExists{parskip.sty}{%
    \usepackage{parskip}
  }{% else
    \setlength{\parindent}{0pt}
    \setlength{\parskip}{6pt plus 2pt minus 1pt}}
}{% if KOMA class
  \KOMAoptions{parskip=half}}
\makeatother
\usepackage{xcolor}
\usepackage[margin=1in]{geometry}
\usepackage{graphicx}
\makeatletter
\def\maxwidth{\ifdim\Gin@nat@width>\linewidth\linewidth\else\Gin@nat@width\fi}
\def\maxheight{\ifdim\Gin@nat@height>\textheight\textheight\else\Gin@nat@height\fi}
\makeatother
% Scale images if necessary, so that they will not overflow the page
% margins by default, and it is still possible to overwrite the defaults
% using explicit options in \includegraphics[width, height, ...]{}
\setkeys{Gin}{width=\maxwidth,height=\maxheight,keepaspectratio}
% Set default figure placement to htbp
\makeatletter
\def\fps@figure{htbp}
\makeatother
\setlength{\emergencystretch}{3em} % prevent overfull lines
\providecommand{\tightlist}{%
  \setlength{\itemsep}{0pt}\setlength{\parskip}{0pt}}
\setcounter{secnumdepth}{5}
\ifLuaTeX
  \usepackage{selnolig}  % disable illegal ligatures
\fi
\usepackage{bookmark}
\IfFileExists{xurl.sty}{\usepackage{xurl}}{} % add URL line breaks if available
\urlstyle{same}
\hypersetup{
  pdftitle={Documentación del Proyecto: Predicción de Precios de Vuelos},
  pdfauthor={Rocío Perez Gregorini y Gonzalo Vazquez},
  hidelinks,
  pdfcreator={LaTeX via pandoc}}

\title{Documentación del Proyecto: Predicción de Precios de Vuelos}
\author{Rocío Perez Gregorini y Gonzalo Vazquez}
\date{2025-12-08}

\begin{document}
\maketitle

{
\setcounter{tocdepth}{2}
\tableofcontents
}
\section{1. Descripción General}\label{descripciuxf3n-general}

\textbf{Objetivo:} Desarrollar un modelo predictivo para estimar tarifas
aéreas en el mercado indio, identificando los factores estructurales
(aerolínea, escalas, duración) que determinan el precio.

\textbf{Hipótesis:} El precio no es estocástico; existen primas de marca
(aerolíneas Full-Service vs.~Low-Cost) y costos operativos por tiempo
que explican la varianza de las tarifas.

\section{2. Estructura del Proyecto}\label{estructura-del-proyecto}

El proyecto respeta el principio de autocontención y rutas relativas
(\texttt{here}):

\begin{itemize}
\tightlist
\item
  \texttt{/config} → Parámetros globales y rutas.
\item
  \texttt{/data} → Datasets:

  \begin{itemize}
  \tightlist
  \item
    \texttt{raw}: Archivo original (\texttt{Data\_Train.xlsx}).
  \item
    \texttt{processed}: Archivos intermedios (\texttt{.rds}).
  \end{itemize}
\item
  \texttt{/functions} → Funciones personalizadas de limpieza y
  visualización.
\item
  \texttt{/scripts} → Código de ejecución secuencial (01 al 05).
\item
  \texttt{/outputs} → Resultados finales:

  \begin{itemize}
  \tightlist
  \item
    \texttt{figures}: Gráficos en alta calidad (\texttt{.png}).
  \item
    \texttt{tables}: Tablas de diagnóstico y regresión (\texttt{.csv}).
  \item
    \texttt{reports}: Gráficos editorializados y conclusiones
    (\texttt{.txt}).
  \end{itemize}
\end{itemize}

\section{3. Flujo de Trabajo (Instrucciones de
Ejecución)}\label{flujo-de-trabajo-instrucciones-de-ejecuciuxf3n}

Para reproducir el análisis, ejecutar los scripts en el siguiente orden:

\begin{enumerate}
\def\labelenumi{\arabic{enumi}.}
\tightlist
\item
  \textbf{\texttt{01\_carga\_datos.R}}: Ingesta del Excel crudo y
  validación de estructura.
\item
  \textbf{\texttt{02\_limpieza.R}}: Procesamiento de fechas, duraciones
  y detección de outliers.
\item
  \textbf{\texttt{03\_exploratorio.R}}: EDA estadístico y visual.
\item
  \textbf{\texttt{04\_analisis\_principal.R}}: Modelado inferencial y
  predictivo.
\item
  \textbf{\texttt{05\_storytelling.R}}: Generación de reportes de
  negocio.
\end{enumerate}

\section{4. Detalle de Scripts}\label{detalle-de-scripts}

\subsubsection{\texorpdfstring{\textbf{01\_carga\_datos.R}}{01\_carga\_datos.R}}\label{carga_datos.r}

\begin{itemize}
\tightlist
\item
  \textbf{Función:} Carga inicial y validación.
\item
  \textbf{Procesos:} Chequeo de dimensiones, existencia de variables
  clave y primer vistazo con \texttt{skimr}.
\item
  \textbf{Output:} \texttt{data/processed/datos\_raw.rds}.
\end{itemize}

\subsubsection{\texorpdfstring{\textbf{02\_limpieza.R}}{02\_limpieza.R}}\label{limpieza.r}

\begin{itemize}
\tightlist
\item
  \textbf{Función:} Higiene de datos.
\item
  \textbf{Procesos:} - Estandarización de columnas a
  \texttt{snake\_case}.

  \begin{itemize}
  \tightlist
  \item
    Conversión de \texttt{Duration} (string) a minutos (numérico).
  \item
    Tratamiento de fechas y horas.
  \item
    Detección de outliers (se decide conservarlos bajo justificación de
    negocio).
  \end{itemize}
\item
  \textbf{Output:} \texttt{data/processed/datos\_clean.rds}.
\end{itemize}

\subsubsection{\texorpdfstring{\textbf{03\_exploratorio.R}
(EDA)}{03\_exploratorio.R (EDA)}}\label{exploratorio.r-eda}

\begin{itemize}
\tightlist
\item
  \textbf{Función:} Entendimiento de distribuciones.
\item
  \textbf{Procesos:} - Histogramas de precios (detección de asimetría
  positiva).

  \begin{itemize}
  \tightlist
  \item
    Boxplots comparativos por aerolínea.
  \item
    Tablas de frecuencias y estadísticas descriptivas.
  \end{itemize}
\item
  \textbf{Output:} Imágenes en \texttt{outputs/figures} y tablas resumen
  en \texttt{outputs/tables}.
\end{itemize}

\subsubsection{\texorpdfstring{\textbf{04\_analisis\_principal.R}
(Modelado)}{04\_analisis\_principal.R (Modelado)}}\label{analisis_principal.r-modelado}

\begin{itemize}
\tightlist
\item
  \textbf{Función:} Inferencia Estadística y Regresión.
\item
  \textbf{Procesos:}

  \begin{itemize}
  \tightlist
  \item
    \textbf{Feature Engineering:} Creación de variables
    \texttt{log\_price}, \texttt{momento\_dia} y segmentación de
    aerolíneas.
  \item
    \textbf{ANOVA \& Tukey:} Validación de diferencias significativas de
    precios entre marcas.
  \item
    \textbf{Regresión Lineal:} Ajuste del modelo predictivo (\(R^2\)
    ajustado \textasciitilde{} 0.65).
  \item
    \textbf{Validación de Supuestos:} Test de Durbin-Watson
    (Independencia) y VIF (Multicolinealidad).
  \end{itemize}
\item
  \textbf{Output:} Gráficos de diagnóstico, coeficientes del modelo y
  performance.
\end{itemize}

\subsubsection{\texorpdfstring{\textbf{05\_storytelling.R} (Reporte
Final)}{05\_storytelling.R (Reporte Final)}}\label{storytelling.r-reporte-final}

\begin{itemize}
\tightlist
\item
  \textbf{Función:} Comunicación de Hallazgos.
\item
  \textbf{Procesos:} Generación de gráficos editorializados para el
  informe final.

  \begin{itemize}
  \tightlist
  \item
    \emph{Gráfico 1:} Ranking de Costo de Marca (Premium vs Low-Cost).
  \item
    \emph{Gráfico 2:} Dinámica de precios por duración.
  \item
    \emph{Dashboard:} Lámina resumen con \texttt{patchwork}.
  \end{itemize}
\item
  \textbf{Output:} Imágenes finales en \texttt{outputs/reports}.
\end{itemize}

\section{5. Librerías Utilizadas}\label{libreruxedas-utilizadas}

El proyecto requiere los siguientes paquetes de R:

\begin{itemize}
\tightlist
\item
  \textbf{Manipulación:} \texttt{tidyverse}, \texttt{dplyr},
  \texttt{janitor}, \texttt{lubridate}.
\item
  \textbf{Gestión de Proyectos:} \texttt{here}, \texttt{readxl},
  \texttt{readr}.
\item
  \textbf{Visualización:} \texttt{ggplot2}, \texttt{scales},
  \texttt{patchwork}, \texttt{ggrepel}.
\item
  \textbf{Estadística/Modelado:} \texttt{broom}, \texttt{car} (VIF),
  \texttt{lmtest} (Durbin-Watson), \texttt{skimr}.
\end{itemize}

\begin{center}\rule{0.5\linewidth}{0.5pt}\end{center}

\emph{Trabajo Final - Ciencia de Datos para Economía y Negocios}

\end{document}
